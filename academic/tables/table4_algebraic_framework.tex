% Table 4: Algebraic Framework Summary
% Author: OTLP Project
% Date: October 20, 2025

\begin{table}[t]
\centering
\caption{Algebraic framework summary showing three complementary algebraic structures for OTLP. Traces form monoids (for composition), span relationships form lattices (for hierarchy), and transformations form categories (for pipeline correctness). Implemented in Haskell (2,800 lines) with 500+ QuickCheck properties verified.}
\label{tab:algebraic-framework}
\small
\begin{tabular}{|l|c|p{4cm}|}
\hline
\textbf{Structure} & \textbf{Operation} & \textbf{Application} \\
\hline
\hline
\multicolumn{3}{|c|}{\textit{Algebraic Structures}} \\
\hline
\makecell[l]{\textbf{Monoid}\\ $(T, \oplus, \varepsilon)$} & \makecell{Trace\\ Composition} & Parallel trace construction, incremental aggregation \\
\hline
\makecell[l]{\textbf{Lattice}\\ $(S, \sqsubseteq, \sqcap, \sqcup)$} & \makecell{LCA $\sqcap$\\ FCD $\sqcup$} & Information flow analysis, span hierarchy reasoning \\
\hline
\makecell[l]{\textbf{Category}\\ $(\mathcal{T}, \circ, id)$} & \makecell{Transformation\\ $\circ$} & Pipeline correctness, verified transformations \\
\hline
\hline
\multicolumn{3}{|c|}{\textit{Proven Properties}} \\
\hline
\multicolumn{2}{|l|}{Associativity: $(t_1 \oplus t_2) \oplus t_3 = t_1 \oplus (t_2 \oplus t_3)$} & Theorem 2 \\
\multicolumn{2}{|l|}{Identity: $t \oplus \varepsilon = \varepsilon \oplus t = t$} & Theorem 2 \\
\multicolumn{2}{|l|}{Partial Order: $\sqsubseteq$ is reflexive, antisymmetric, transitive} & Theorem 3 \\
\multicolumn{2}{|l|}{Meet/Join: $\sqcap$ (LCA) and $\sqcup$ (FCD) exist} & Theorem 3 \\
\multicolumn{2}{|l|}{Category Laws: Associativity + Identity} & Theorem 4 \\
\hline
\hline
\multicolumn{3}{|c|}{\textit{Correctness Theorems}} \\
\hline
\multicolumn{2}{|l|}{\textbf{Theorem 5}: Composition preserves validity} & Structural Induction \\
\multicolumn{2}{|l|}{\textbf{Theorem 6}: Transformation correctness} & Property Preservation \\
\hline
\end{tabular}

\vspace{0.2cm}

\textbf{Implementation \& Verification:}
\begin{itemize}[leftmargin=*, noitemsep, topsep=2pt]
    \item \textbf{Haskell Implementation}: 2,800 lines with type-safe algebraic operations
    \item \textbf{QuickCheck Verification}: 500+ properties verified (monoid laws, lattice properties, category laws, functors, natural transformations)
    \item \textbf{Practical Benefits}: Enables compositional reasoning, hierarchical analysis, and transformation verification in OpenTelemetry pipelines
\end{itemize}

\vspace{0.2cm}

\textbf{Example Applications:}
\begin{itemize}[leftmargin=*, noitemsep, topsep=2pt]
    \item \textbf{Monoid}: Merging traces from different services/threads in parallel
    \item \textbf{Lattice}: Finding lowest common ancestor (LCA) for span fork points
    \item \textbf{Category}: Verifying correctness of Collector processor pipelines
\end{itemize}
\end{table}

