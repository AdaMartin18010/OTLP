% Figure 6: Triple Flow Analysis Framework
% Purpose: Show control/data/execution flow analysis
% Location: Section 3.4 (Flow Analysis)
% Size: Double column width (figure*)

\begin{figure*}[t]
\centering
\begin{tikzpicture}[
    node distance=0.8cm and 1.2cm,
    func/.style={
        rectangle, rounded corners=3pt,
        draw=black, thick,
        minimum width=2cm,
        minimum height=0.7cm,
        text centered,
        font=\footnotesize,
        fill=blue!20
    },
    data/.style={
        ellipse,
        draw=black, thick,
        minimum width=1.5cm,
        minimum height=0.6cm,
        text centered,
        font=\footnotesize,
        fill=yellow!30
    },
    exec/.style={
        rectangle, rounded corners=2pt,
        draw=black, thick,
        minimum width=2cm,
        minimum height=0.6cm,
        text centered,
        font=\footnotesize,
        fill=green!20
    },
    cflow/.style={->, very thick, blue, >=stealth},
    dflow/.style={->, thick, dashed, orange, >=stealth},
    eflow/.style={->, thick, dotted, green!60!black, >=stealth, line width=1.5pt}
]

% Title labels for three columns
\node[font=\small\bfseries, text=blue] at (-5, 0) {Control Flow};
\node[font=\small\bfseries, text=orange] at (0, 0) {Data Flow};
\node[font=\small\bfseries, text=green!60!black] at (5, 0) {Execution Flow};

% ===== Control Flow Column =====
\node[func, below=0.5cm of {(-5,0)}] (cf-start) {\texttt{startSpan()}};
\node[func, below=of cf-start] (cf-process) {\texttt{processReq()}};
\node[func, below left=of cf-process] (cf-db) {\texttt{queryDB()}};
\node[func, below right=of cf-process] (cf-cache) {\texttt{checkCache()}};
\node[func, below=2cm of cf-process] (cf-end) {\texttt{endSpan()}};

% Control flow arrows
\draw[cflow] (cf-start) -- (cf-process);
\draw[cflow] (cf-process) -- (cf-db);
\draw[cflow] (cf-process) -- (cf-cache);
\draw[cflow] (cf-db) -- (cf-end);
\draw[cflow] (cf-cache) -- (cf-end);

% ===== Data Flow Column =====
\node[data, below=0.5cm of {(0,0)}] (df-ctx) {TraceCtx};
\node[data, below=of df-ctx] (df-span) {Span};
\node[data, below left=of df-span] (df-attr1) {Attributes};
\node[data, below right=of df-span] (df-attr2) {Events};
\node[data, below=2cm of df-span] (df-result) {TraceData};

% Data flow arrows
\draw[dflow] (df-ctx) -- (df-span) node[midway, right, font=\tiny] {context};
\draw[dflow] (df-span) -- (df-attr1) node[midway, left, font=\tiny] {add};
\draw[dflow] (df-span) -- (df-attr2) node[midway, right, font=\tiny] {record};
\draw[dflow] (df-attr1) -- (df-result);
\draw[dflow] (df-attr2) -- (df-result);

% ===== Execution Flow Column =====
\node[exec, below=0.5cm of {(5,0)}] (ef-t0) {$t_0$: Span created};
\node[exec, below=of ef-t0] (ef-t1) {$t_1$: Processing};
\node[exec, below left=of ef-t1] (ef-t2) {$t_2$: DB query};
\node[exec, below right=of ef-t1] (ef-t3) {$t_3$: Cache hit};
\node[exec, below=2cm of ef-t1] (ef-t4) {$t_4$: Span closed};

% Execution flow arrows
\draw[eflow] (ef-t0) -- (ef-t1) node[midway, right, font=\tiny] {10ms};
\draw[eflow] (ef-t1) -- (ef-t2) node[midway, left, font=\tiny] {15ms};
\draw[eflow] (ef-t1) -- (ef-t3) node[midway, right, font=\tiny] {12ms};
\draw[eflow] (ef-t2) -- (ef-t4) node[midway, left, font=\tiny] {50ms};
\draw[eflow] (ef-t3) -- (ef-t4) node[midway, right, font=\tiny] {5ms};

% ===== Cross-flow relationships (gray dashed) =====
% Control to Data
\draw[->, thick, dashed, gray] (cf-start.east) to[bend left=15] (df-ctx.west);
\draw[->, thick, dashed, gray] (cf-process.east) to[bend left=10] (df-span.west);

% Data to Execution
\draw[->, thick, dashed, gray] (df-span.east) to[bend right=10] (ef-t1.west);
\draw[->, thick, dashed, gray] (df-result.east) to[bend right=15] (ef-t4.west);

% Control to Execution
\draw[->, thick, dashed, gray] (cf-db.east) to[bend left=20] (ef-t2.west);
\draw[->, thick, dashed, gray] (cf-cache.east) to[bend left=20] (ef-t3.west);

% Bottom legend
\node[draw=black, thick, rectangle, rounded corners=3pt,
      below=1.2cm of df-result, minimum width=10cm, minimum height=1.8cm,
      fill=white] (legend) {};
\node[font=\small\bfseries, above=0.05cm of legend.north] {Verification Checks};

\node[font=\tiny, align=left] at ([xshift=-3cm]legend.center) {
    \textcolor{blue}{\textbf{Control Flow:}}\\
    • Call graph correctness\\
    • Span hierarchy matches\\
    • No orphaned spans
};

\node[font=\tiny, align=left] at (legend.center) {
    \textcolor{orange}{\textbf{Data Flow:}}\\
    • Context propagation\\
    • Attribute consistency\\
    • No data loss
};

\node[font=\tiny, align=left] at ([xshift=3cm]legend.center) {
    \textcolor{green!60!black}{\textbf{Execution Flow:}}\\
    • Causality preservation\\
    • Timestamp ordering\\
    • Duration validity
};

\end{tikzpicture}
\caption{Triple Flow Analysis Framework. Our framework analyzes three complementary aspects of OTLP traces: (1) \textbf{Control flow} (blue, solid) verifies that span relationships match the program's call graph structure; (2) \textbf{Data flow} (orange, dashed) tracks context and attribute propagation to detect information loss; (3) \textbf{Execution flow} (green, dotted) checks temporal properties including causality and timestamp ordering. Gray dashed lines show cross-flow dependencies. By combining all three analyses, we achieve comprehensive verification coverage.}
\label{fig:triple-flow}
\end{figure*}

