% Figure 4: Span Composition Monoid
% Purpose: Illustrate monoid properties for span composition
% Location: Section 3.3 (Algebraic Structures)
% Size: Single column width

\begin{figure}[t]
\centering
\begin{tikzpicture}[
    node distance=1.5cm,
    span/.style={
        rectangle, rounded corners=3pt,
        draw=black, thick,
        minimum width=2.5cm,
        minimum height=0.8cm,
        text centered,
        font=\footnotesize,
        fill=blue!20
    },
    identity/.style={
        rectangle, rounded corners=3pt,
        draw=black, thick,
        minimum width=1.8cm,
        minimum height=0.6cm,
        text centered,
        font=\footnotesize,
        fill=gray!20
    },
    operation/.style={
        circle,
        draw=black,
        thick,
        minimum size=0.8cm,
        text centered,
        font=\large,
        fill=yellow!30
    }
]

% Top: Associativity demonstration
\node[font=\small\bfseries] at (0, 0) {Associativity: $(s_1 \oplus s_2) \oplus s_3 = s_1 \oplus (s_2 \oplus s_3)$};

% Left side: (s1 ⊕ s2) ⊕ s3
\node[span, below=0.8cm of {(0,0)}, xshift=-3.5cm] (s1-left) {$s_1$\\[2pt]{\tiny [t1, t2]}};
\node[operation, right=0.3cm of s1-left] (op1-left) {$\oplus$};
\node[span, right=0.3cm of op1-left] (s2-left) {$s_2$\\[2pt]{\tiny [t2, t3]}};
\node[operation, right=0.3cm of s2-left] (op2-left) {$\oplus$};
\node[span, right=0.3cm of op2-left] (s3-left) {$s_3$\\[2pt]{\tiny [t3, t4]}};

% Brackets for left grouping
\draw[thick, gray] ([yshift=-0.5cm]s1-left.south west) -- ([yshift=-0.5cm]s1-left.south east);
\draw[thick, gray] ([yshift=-0.5cm]s1-left.south west) -- ([yshift=-0.7cm]s1-left.south west);
\draw[thick, gray] ([yshift=-0.5cm]s2-left.south east) -- ([yshift=-0.5cm]s2-left.south west);
\draw[thick, gray] ([yshift=-0.5cm]s2-left.south east) -- ([yshift=-0.7cm]s2-left.south east);

% Right side: s1 ⊕ (s2 ⊕ s3)
\node[span, below=0.8cm of {(0,0)}, xshift=3.5cm] (s1-right) {$s_1$\\[2pt]{\tiny [t1, t2]}};
\node[operation, right=0.3cm of s1-right] (op1-right) {$\oplus$};
\node[span, right=0.3cm of op1-right] (s2-right) {$s_2$\\[2pt]{\tiny [t2, t3]}};
\node[operation, right=0.3cm of s2-right] (op2-right) {$\oplus$};
\node[span, right=0.3cm of op2-right] (s3-right) {$s_3$\\[2pt]{\tiny [t3, t4]}};

% Brackets for right grouping
\draw[thick, gray] ([yshift=-0.5cm]s2-right.south west) -- ([yshift=-0.5cm]s2-right.south east);
\draw[thick, gray] ([yshift=-0.5cm]s2-right.south west) -- ([yshift=-0.7cm]s2-right.south west);
\draw[thick, gray] ([yshift=-0.5cm]s3-right.south east) -- ([yshift=-0.5cm]s3-right.south west);
\draw[thick, gray] ([yshift=-0.5cm]s3-right.south east) -- ([yshift=-0.7cm]s3-right.south east);

% Result for both
\node[span, below=1.5cm of s2-left, xshift=-3cm, minimum width=3.5cm] (result-left) {$s_1 \oplus s_2 \oplus s_3$\\[2pt]{\tiny [t1, t4]}};
\node[span, below=1.5cm of s2-right, xshift=3cm, minimum width=3.5cm] (result-right) {$s_1 \oplus s_2 \oplus s_3$\\[2pt]{\tiny [t1, t4]}};

\draw[->, very thick] ([yshift=-0.8cm]s2-left.south) -- (result-left.north);
\draw[->, very thick] ([yshift=-0.8cm]s2-right.south) -- (result-right.north);

% Equals sign
\node[font=\Large] at (0, -3.8) {$=$};

% Bottom: Identity element
\node[font=\small\bfseries, below=3.5cm of result-left] (id-title) {Identity: $s \oplus \varepsilon = \varepsilon \oplus s = s$};

\node[span, below=0.5cm of id-title, xshift=-2.5cm] (s-id) {$s$\\[2pt]{\tiny [t1, t2]}};
\node[operation, right=0.3cm of s-id] (op-id) {$\oplus$};
\node[identity, right=0.3cm of op-id] (epsilon) {$\varepsilon$\\[2pt]{\tiny (empty)}};
\node[font=\Large, right=0.5cm of epsilon] (eq-id) {$=$};
\node[span, right=0.5cm of eq-id] (s-result) {$s$\\[2pt]{\tiny [t1, t2]}};

\end{tikzpicture}
\caption{Span Composition Monoid $(S, \oplus, \varepsilon)$. Top: Associativity property ensures that sequential composition of spans is order-independent when grouped. Bottom: Identity element $\varepsilon$ (empty span) acts as a neutral element. These properties enable compositional reasoning about trace construction and allow parallel span aggregation without coordination.}
\label{fig:monoid-composition}
\end{figure}

