% Figure 3: Type System Hierarchy
% Purpose: Show the hierarchical structure of OTLP types
% Location: Section 3.2 (Type System)
% Size: Single column width

\begin{figure}[t]
\centering
\begin{tikzpicture}[
    level 1/.style={sibling distance=4cm, level distance=1.2cm},
    level 2/.style={sibling distance=2cm, level distance=1.0cm},
    level 3/.style={sibling distance=1.5cm, level distance=0.9cm},
    every node/.style={
        rectangle, rounded corners=2pt,
        draw=black, thick,
        minimum width=2.2cm,
        minimum height=0.6cm,
        text centered,
        font=\footnotesize
    },
    base/.style={fill=gray!20},
    refined/.style={fill=blue!20},
    dependent/.style={fill=green!20},
    edge from parent/.style={->, thick, >=stealth}
]

% Root: Base Types
\node[base] {Base Types}
    % Level 1: Primitive and Structured
    child {
        node[base] {Primitive}
        % Level 2: Specific primitive types
        child {node[base] {Int}}
        child {node[base] {String}}
        child {node[base] {Timestamp}}
    }
    child {
        node[base] {Structured}
        % Level 2: Complex structures
        child {node[base] {Span}}
        child {node[base] {Trace}}
        child {node[base] {Resource}}
    };

% Refinement Types (positioned to the right)
\node[refined, right=5.5cm of {(0,0)}] (ref-root) {Refinement Types}
    child {
        node[refined] {$\{x: \text{Int} \mid P(x)\}$}
        child[level distance=0.8cm] {node[refined, font=\tiny] {StatusCode\\$\{x \mid 100 \leq x \leq 599\}$}}
    }
    child {
        node[refined] {Non-empty}
        child[level distance=0.8cm] {node[refined, font=\tiny] {TraceID\\$\{x \mid |x| = 16\}$}}
    };

% Dependent Types (positioned below)
\node[dependent, below=4.5cm of {(0,0)}, xshift=2.5cm] (dep-root) {Dependent Types}
    child[level distance=1.0cm] {
        node[dependent, font=\tiny] {$\text{Span}[p:\text{ParentID}]$}
    }
    child[level distance=1.0cm] {
        node[dependent, font=\tiny] {$\text{Trace}[n:\mathbb{N}]$}
    };

% Connections showing type relationships
\draw[->, thick, dashed, blue] ([yshift=-0.3cm]base.south) 
    to[out=-45, in=180] 
    node[midway, above, font=\tiny, sloped] {refines} 
    ([yshift=0.5cm]ref-root.west);

\draw[->, thick, dashed, green!60!black] ([yshift=-0.3cm]base.south) 
    to[out=-90, in=90] 
    node[midway, right, font=\tiny] {depends on} 
    (dep-root.north);

% Legend
\node[base, below=5.3cm of {(0,0)}, xshift=-2.5cm, minimum width=1.2cm, minimum height=0.4cm] (leg-base) {};
\node[font=\tiny, right=0.1cm of leg-base] {Base};

\node[refined, right=0.3cm of leg-base, minimum width=1.2cm, minimum height=0.4cm] (leg-ref) {};
\node[font=\tiny, right=0.1cm of leg-ref] {Refined};

\node[dependent, right=0.3cm of leg-ref, minimum width=1.2cm, minimum height=0.4cm] (leg-dep) {};
\node[font=\tiny, right=0.1cm of leg-dep] {Dependent};

\end{tikzpicture}
\caption{OTLP Type System Hierarchy. Base types (gray) provide fundamental structures like Span and Trace. Refinement types (blue) add predicates to constrain values (e.g., HTTP status codes must be in [100,599]). Dependent types (green) capture relationships between entities (e.g., a Span depends on its parent's existence). Our type system combines all three approaches to ensure structural correctness.}
\label{fig:type-hierarchy}
\end{figure}

