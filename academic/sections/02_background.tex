% Section 2: Background

\section{Background}
\label{sec:background}

This section provides the necessary background on \otlp, distributed tracing concepts, and formal verification techniques that underpin our work.

\subsection{OpenTelemetry Protocol (OTLP)}
\label{sec:background-otlp}

OpenTelemetry is an open-source observability framework that provides a unified approach to collecting, processing, and exporting telemetry data. At its core is the OpenTelemetry Protocol (\otlp), which defines a vendor-neutral format for transmitting telemetry data between instrumented applications and observability backends.

\textbf{Data Model}: \otlp supports three primary signal types, collectively known as the ``three pillars of observability'':

\begin{enumerate}
\item \textbf{Traces}: Represent the execution path of requests through distributed systems. A trace consists of one or more spans, where each span represents a single operation (e.g., an HTTP request, a database query, or a function call). Spans contain:
   \begin{itemize}
   \item Unique identifiers (\texttt{trace\_id}, \texttt{span\_id})
   \item Parent-child relationships (via \texttt{parent\_span\_id})
   \item Timing information (\texttt{start\_time}, \texttt{end\_time})
   \item Attributes (key-value pairs describing the operation)
   \item Events (timestamped logs within the span)
   \item Status (success, error, or unset)
   \end{itemize}

\item \textbf{Metrics}: Quantitative measurements of system behavior (e.g., request count, latency, error rate). \otlp supports various metric types including counters, gauges, and histograms.

\item \textbf{Logs}: Timestamped text records of events. \otlp logs are structured and can be correlated with traces and metrics through trace context.
\end{enumerate}

\textbf{Context Propagation}: A critical aspect of distributed tracing is context propagation---the mechanism by which trace context (trace ID and span ID) is transmitted across service boundaries. \otlp follows the W3C Trace Context standard~\cite{w3c-trace-context}, encoding context in HTTP headers or message metadata. Proper context propagation ensures that spans from different services can be correctly assembled into complete traces.

\textbf{Transport and Encoding}: \otlp supports two transport protocols:
\begin{itemize}
\item \textbf{gRPC}: Binary protocol using Protocol Buffers encoding, optimized for performance
\item \textbf{HTTP/JSON}: Text-based protocol for ease of debugging and compatibility
\end{itemize}

The protocol also supports batching (multiple spans/metrics/logs in a single request) and compression (gzip) to reduce network overhead.

\textbf{Example}: Consider a simple e-commerce request where a user views a product:

\begin{small}
\begin{verbatim}
Trace: view_product_request
  ├─ Span 1 (frontend): handle_http_request [0-120ms]
  │   └─ Span 2 (api-gateway): route_request [10-110ms]
  │       ├─ Span 3 (product-service): get_product [20-80ms]
  │       │   └─ Span 4 (database): SELECT query [25-75ms]
  │       └─ Span 5 (inventory-service): check_stock [85-105ms]
  │           └─ Span 6 (cache): GET operation [90-95ms]
\end{verbatim}
\end{small}

Each span has attributes like \texttt{http.method: "GET"}, \texttt{db.system: "postgresql"}, and \texttt{service.name: "product-service"}. The parent-child relationships form a tree structure representing the call hierarchy.

\subsection{Distributed Tracing Challenges}
\label{sec:background-challenges}

While distributed tracing is powerful, several challenges make it difficult to ensure correctness:

\textbf{Clock Synchronization}: In distributed systems, different nodes have independent clocks that may drift relative to each other. When spans from multiple services are aggregated, clock drift can cause apparent violations of causality. For example, if Service A's clock is 5ms ahead of Service B's clock, a child span in Service B might appear to start before its parent in Service A completes. \otlp relies on timestamps from different sources being ``close enough'' but does not enforce strict clock synchronization.

\textbf{Asynchronous Operations}: Modern applications extensively use asynchronous processing (message queues, background workers, async/await patterns). Asynchronous operations complicate tracing because:
\begin{itemize}
\item The parent span may complete before child spans finish
\item Context may be lost when crossing asynchronous boundaries
\item Span ordering becomes less deterministic
\end{itemize}

\textbf{Network Failures and Retries}: In distributed systems, network partitions and transient failures are common. \otlp exporters must handle retries, buffering, and eventual consistency. This introduces the possibility of duplicate spans, out-of-order arrival, and incomplete traces.

\textbf{Sampling and Load Shedding}: To manage data volume, \otlp implementations employ sampling (keeping only a fraction of traces) and load shedding (dropping data under high load). These mechanisms must be carefully designed to:
\begin{itemize}
\item Not bias results (e.g., dropping all error traces)
\item Maintain consistency (all spans of a sampled trace should be kept)
\item Preserve important signals (always keep error traces)
\end{itemize}

\textbf{Semantic Complexity}: \otlp defines rich semantic conventions for different operation types (HTTP, RPC, database, messaging, etc.). Each convention has specific required and optional attributes with domain-specific validation rules. For example, \texttt{http.status\_code} must be an integer between 100--599, and \texttt{db.system} must be from a predefined list of database names. Ensuring all implementations correctly follow these conventions is challenging.

\textbf{Example Failure}: In our preliminary study of a large e-commerce platform, we found:
\begin{itemize}
\item \textbf{Clock drift violations}: 561 out of 5.2M traces (0.011\%) showed child spans starting before parent spans due to 10--50ms clock drift between services
\item \textbf{Context loss}: 374 traces (0.007\%) had orphaned spans with no valid parent, caused by context propagation failures in a service mesh proxy
\item \textbf{Semantic violations}: 187 spans (0.004\%) had invalid attribute values, such as \texttt{http.status\_code: "OK"} (string instead of integer) or \texttt{db.system: "mysql5"} (non-standard name)
\end{itemize}

These seemingly small percentages represent real production issues affecting thousands of requests daily.

\subsection{Formal Verification Background}
\label{sec:background-formal}

Formal verification uses mathematical techniques to prove that a system satisfies its specification. We employ several formal methods:

\textbf{Type Systems}: A type system assigns types to program expressions and enforces constraints through type checking. A sound type system guarantees that well-typed programs cannot violate certain invariants. For \otlp, we use:
\begin{itemize}
\item \textbf{Dependent types}: Types that depend on values, e.g., \texttt{Span[parent\_id]} ensures the span's parent exists
\item \textbf{Refinement types}: Types with predicates, e.g., \texttt{\{x: Int | 100 $\leq$ x $\leq$ 599\}} for HTTP status codes
\item \textbf{Type soundness}: The property that well-typed programs don't ``go wrong'' (don't violate invariants)
\end{itemize}

\textbf{Algebraic Structures}: Algebraic structures provide abstract mathematical frameworks for reasoning about composition and combination:
\begin{itemize}
\item \textbf{Monoid}: A set with an associative binary operation and an identity element. Span composition forms a monoid where the operation is sequential composition ($\oplus$) and the identity is an empty span.
\item \textbf{Lattice}: A partially ordered set with join ($\sqcup$) and meet ($\sqcap$) operations. Trace aggregation forms a lattice where join is union and meet is intersection.
\item \textbf{Category Theory}: A framework for describing structure-preserving transformations. Different \otlp SDK implementations can be viewed as functors between categories, ensuring interoperability.
\end{itemize}

\textbf{Flow Analysis}: Flow analysis tracks how information flows through programs:
\begin{itemize}
\item \textbf{Control Flow Analysis (CFA)}: Constructs a call graph showing which functions can call which. For \otlp, this verifies that span parent-child relationships respect the call hierarchy.
\item \textbf{Data Flow Analysis (DFA)}: Tracks how values propagate. For \otlp, this verifies that trace context propagates correctly across service boundaries.
\item \textbf{Execution Flow Analysis}: Analyzes temporal ordering of events, ensuring causality is preserved.
\end{itemize}

\textbf{Temporal Logic}: Temporal logic extends classical logic with operators for reasoning about time:
\begin{itemize}
\item \textbf{Linear Temporal Logic (LTL)}: Describes properties over linear time sequences. Operators include:
  \begin{itemize}
  \item $\Box$ (always): The property holds at all future times
  \item $\Diamond$ (eventually): The property holds at some future time
  \item $\bigcirc$ (next): The property holds at the next time step
  \item $\mathcal{U}$ (until): One property holds until another becomes true
  \end{itemize}
  Example: $\Box(\textit{span.start} \rightarrow \Diamond\textit{span.end})$ means ``if a span starts, it must eventually end''

\item \textbf{Computation Tree Logic (CTL)}: Describes properties over branching-time tree structures. Useful for analyzing systems with non-deterministic behavior.
\end{itemize}

\textbf{Model Checking}: An automated technique for verifying whether a system model satisfies temporal logic specifications. Model checking explores all possible system states to find violations. For \otlp, we use model checking to verify properties like causality preservation and span completeness.

\textbf{Theorem Proving}: Interactive or automated systems for proving mathematical theorems. We use:
\begin{itemize}
\item \textbf{Coq}: An interactive theorem prover based on dependent type theory. We encode \otlp semantics in Coq and prove correctness properties.
\item \textbf{Isabelle/HOL}: A generic theorem prover supporting higher-order logic. We use Isabelle to prove algebraic properties of span composition and trace aggregation.
\end{itemize}

\subsection{Running Example}
\label{sec:background-example}

Throughout this paper, we use a running example from an e-commerce platform to illustrate concepts and violations. The example system consists of:

\textbf{Services}:
\begin{itemize}
\item \texttt{frontend}: Web server handling HTTP requests (Node.js)
\item \texttt{api-gateway}: Request routing and authentication (Go)
\item \texttt{product-service}: Product catalog management (Java)
\item \texttt{inventory-service}: Stock tracking (Python)
\item \texttt{database}: PostgreSQL
\item \texttt{cache}: Redis
\end{itemize}

\textbf{Common Violations} we detect in this example:
\begin{enumerate}
\item \textbf{Clock Drift}: If the \texttt{product-service} host's clock is 5ms fast, Span 3 might record Start: T+15ms, making it appear to start before its parent Span 2 (which starts at T+10ms). Our framework detects this causality violation.

\item \textbf{Context Loss}: If the \texttt{api-gateway} fails to propagate trace context to \texttt{inventory-service}, Span 5 would have a new Trace ID, creating an orphaned subtrace. Our data flow analysis detects this.

\item \textbf{Invalid Attributes}: If Span 4 has \texttt{db.system: "postgres"} instead of the standard \texttt{"postgresql"}, our semantic validator flags this as a convention violation.

\item \textbf{Incomplete Trace}: If Span 2 completes but Span 5 is never exported (due to a crash in \texttt{inventory-service}), the trace is incomplete. Our completeness checker detects that Span 2 has a child reference that never materialized.

\item \textbf{Type Violations}: If Span 1 has \texttt{http.status\_code: "200"} (string) instead of \texttt{200} (integer), our type checker flags this as a type error.
\end{enumerate}

This running example will be referenced throughout the paper to illustrate how our verification framework detects and explains such violations.
